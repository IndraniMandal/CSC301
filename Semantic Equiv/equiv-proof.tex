%template1.tex
%The following LaTeX source file represents the simplest kind of slide presentation; no overlays, no included graphics. Substitute your favorite style for ``pascal''. To create the PDF file template1.pdf, (1) be sure to use the prosper class, then (2) execute the command latex template1.tex, and (3) the command dvipdf template1.dvi.

%%%%%%%%%%%%%%%%%%%%%%%%%%%%%%% template1.tex %%%%%%%%%%%%%%%%%%%%%%%%%%%%%%%%%%%
\documentclass[a4paper,blends,pdf,colorBG,slideColor]{prosper}
% definitions for slides for the semantics course - CSC501
% Lutz Hamel, (c) 2006

\hypersetup{pdfpagemode=FullScreen}

\usepackage{bussproofs}
\usepackage{amssymb}
\usepackage{latexsym}
%\usepackage[usenames]{color}
%\usepackage{xypic}

\newcommand{\orbar}{\;|\;}
\newcommand{\bs}[1]{\begin{slide}{#1}\ptsize{8}}
\newcommand{\es}{\end{slide}}
\newcommand{\co}{\,\colon\;}
\newcommand{\syntaxset}[1]{\ensuremath{\mbox{\bf  #1}}}
\newcommand{\ifstmt}[3]{\ensuremath{{\bf if}\; {#1}\;{\bf then}\;{#2}\;{\bf else}\;{#3}}}
\newcommand{\whilestmt}[2]{\ensuremath{{\bf while}\; {#1}\;{\bf do}\;{#2}}}
\newcommand{\pairmap}[3]{\ensuremath{\langle {#1}, {#2} \rangle \rightarrow {#3}}}
\newcommand{\pair}[2]{\ensuremath{\langle {#1}, {#2} \rangle}}
\newcommand{\cond}[3]{\ensuremath{({#1}?\,{#2} : {#3})}}
\newcommand{\condbar}[3]{\ensuremath{({#1}\rightarrow{#2} \mid {#3})}}
\newcommand{\liftfunc}[1]{\ensuremath{\lfloor{#1}\rfloor}}





\begin{document}

\bs{Equivalence Proof}
{\bf Proposition:} In  language THREE
 the sentence $E + E$ is semantically
equivalent to $2 * E$ for any subexpression $E\in L(\mbox{THREE})$.

(Observe that $E + E \ne 2 *E$ because this is just syntax, that is,
the abstract syntax trees of these two expressions do not look the same, but we 
expect that semantically they express the  same thing.)

{\bf Proof:} From elementary algebra we know that given any integer value $i$ we
have $i + i = 2 \times i$.  We will use this identity in our proof.

Assume that $\langle E, C \rangle \rightarrow e$, that is, we assume that our 
subexpression $E$ evaluates to some integer value $e$ in the context of the
binding environment $C$. So,
\begin{prooftree}
\AxiomC{$\langle {\tt var}(x), C \rangle \rightarrow \mathit{lookup}(x,C)=k$}
\AxiomC{$\langle {\tt var}(x), C \rangle \rightarrow \mathit{lookup}(x,C)=k$}
\BinaryInfC{$\langle {\tt plus}({\tt var}(x),{\tt var}(x)),C\rangle \rightarrow k + k = 2\times k$}
\end{prooftree}
which follows from our algebraic identity above. Now,
\begin{prooftree}
\AxiomC{$\langle 2,C\rangle \rightarrow 2$}
\AxiomC{$\langle {\tt var}(x), C \rangle \rightarrow \mathit{lookup}(x,C)=k$}
\BinaryInfC{$\langle {\tt times}(2,{\tt var}(x)),C\rangle \rightarrow 2 \times k$}
\end{prooftree}
This concludes our proof. $\Box$



\es
\end{document}
%%%%%%%%%%%%%%%%%%%%%%%%%%% end of template1.tex %%%%%%%%%%%%%%%%%%%%%%%%%%%%%%%%

